\newcommand{\Cross}{\mathbin{\tikz [x=1.4ex,y=1.4ex,line width=.2ex] \draw (0,0) -- (1,1) (0,1) -- (1,0);}}%
%\renewcommand{\theequation}{\theenumi}
%\begin{enumerate}[label=\thesection.\arabic*.,ref=\thesection.\theenumi]
%\numberwithin{equation}{enumi}
%	
%	
%
%\item $\triangle AMC \cong \triangle DMB$  by SAS congruency $\because$
%\begin{enumerate}
%\item $AM = BM$
%\item $CM = DM$
%\item $\phase{AMC}$ = $\phase{DMB}$ ( Vertically Opposite Angles)
%\end{enumerate}
%%
%\item From \eqref{eq:constr_b}, \eqref{eq:constr_c} and \eqref{eq:constr_d},
%%
%%
%\begin{align}
%\brak{\vec{D}-\vec{B}}^T
%\brak{\vec{B}-\vec{C}} &= \myvec{0 & b}\myvec{a \\ 0} = 0
%\\
%\implies BD \perp BC
%\end{align}
%%
%\item From \eqref{eq:constr_a}, \eqref{eq:constr_b}, \eqref{eq:constr_c} and \eqref{eq:constr_d},
%\begin{align}
%\norm{\vec{A}-\vec{B}} &= \norm{\myvec{-a \\ b}}
%\\
%\norm{\vec{C}-\vec{D}} &= \norm{\myvec{-a \\ -b}}
%\\
%\implies \norm{\vec{A}-\vec{B}} &= \norm{\vec{C}-\vec{D}}\\
%\text{or, } AB &=CD
%\label{eq:solution_abcd}
%\end{align}
%%
%From RHS congruence,  $\triangle ACB \cong  \triangle DCB$.
%\item From \eqref{eq:solution_abcd}, noting that $\vec{M}$ is the mid point of both $AB$ and $CD$, 
%\begin{align}
%CM = \frac{1}{2}CD =\frac{1}{2} AB
%\end{align}
%
%
%
%\end{enumerate}


%\begin{equation}
%\begin{aligned}
%    Area = \frac{1}{2}\begin{vmatrix}
%      4       & -6    & 1 \\ 
%      3       & -2    & 1 \\
%      5       & 2     & 1 \\
%      
%    \end{vmatrix} \xleftrightarrow[]{C_2\leftarrow \frac{C_2}{2}} \frac{2}{2}\begin{vmatrix}
%      4       &-3    & 1 \\ 
%      3       &-1    & 1 \\
%      5       & 1    & 1 \\
%      
%    \end{vmatrix}\\
%    \xleftrightarrow[R_3 \leftarrow R_3-R_1]{R_2\leftarrow R_2-R_1}
%    \begin{vmatrix}
%      4       & -3    & 1 \\ 
%      -1       & 2    & 0 \\
%      1       & 4   & 0 \\
%      
%    \end{vmatrix} \\ \xleftrightarrow[]{R_3\leftarrow R_3+R_2}
%    \begin{vmatrix}
%      4       & -3    & 1 \\ 
%      -1       & 2    & 0 \\
%      0       & 6   & 0 \\
%      
%    \end{vmatrix} \\ \xleftrightarrow[]{R_3\leftarrow \frac{R_3}{6}}
%    6 \begin{vmatrix}
%      4       & -3    & 1 \\ 
%      -1       & 2    & 0 \\
%      0       & 1   & 0 \\
%      
%    \end{vmatrix} \\
%    =-6
%   \end{aligned}
%  \end{equation}
%$\vec{s}$
The area of a parallelogram can be defined as:\\
$Area$ = $\abs{\vec{a} \Cross \vec{b}} $

%\begin{equation}
%\begin{aligned}
%    \vec{a} \Cross \vec{b} = \begin{vmatrix}
%      $\^i$       & $\^j$    & $\^k$ \\ 
%      3       & 1    & 4 \\
%      1       & -1     & 1 \\
%      
%    \end{vmatrix}
%\end{aligned}
%\end{equation}
%So, $\vec{a} \Cross \vec{b}$ = 5\^i + \^j - 4\^k.\\
%
%
%Now, $\abs{\vec{a} \Cross \vec{b}} $ = $\sqrt{5^2 + 1^2 + (-1)^2}$\\
%\text{or, } $\abs{\vec{a} \Cross \vec{b}} $ = $\sqrt{27}$\\
%\text{or, } $Area$ = $\sqrt{27}$\\
%Hence, the area of the parallelogram in the above problem statement is $\sqrt{27}$.\\
%\text{or, }\\
The cross-product can be calculated as:\\
$\vec{a} \Cross \vec{b} $ = $[\vec{a}]_x \vec{b}$
where $[\vec{a}]_x$ = $\vec{a} \Cross $ \^e 
and \^e is the unit vector. \\
If $\vec{a}$ can be expressed as:\\
\begin{equation}
\begin{aligned}
      \vec{a} = \begin{pmatrix}
      a_1 \\ 
      a_2 \\
      a_3 \\
      
    \end{pmatrix} = \begin{pmatrix}
      3 \\ 
      1 \\
      4 \\
      
    \end{pmatrix}
\end{aligned}
\end{equation} and 
\begin{equation}
\begin{aligned}
      \vec{b} = \begin{pmatrix}
      b_1 \\ 
      b_2 \\
      b_3 \\
      
    \end{pmatrix} = \begin{pmatrix}
      1 \\ 
      -1 \\
      1 \\
      
    \end{pmatrix}
\end{aligned}
\end{equation}
Then $[\vec{a}]_x$ can be expressed as:\\
%$[\vec{a}]_x$ = $ [\vec{a} \Cross $  \^i $\quad$  $\vec{a} \Cross $ \^j $\quad$ $\vec{a} \Cross $ \^k] 
\begin{equation}
\begin{aligned}
       \text{or, }[\vec{a}]_x = \begin{pmatrix}
      0       & -a_3    & a_2 \\ 
      a_3       & 0    & -a_1 \\
      -a_2       & a_1     & 0 \\
      
    \end{pmatrix}= \begin{pmatrix}
      0       & -4    & 1 \\ 
      4       & 0    & -3 \\
      -1       & 3     & 0 \\
      
    \end{pmatrix}
\end{aligned}
\end{equation}
So, the $[\vec{a}]_x \vec{b}$ can be calculated as:\\
\begin{equation}
\begin{aligned}
      [\vec{a}]_x \vec{b} = \begin{pmatrix}
      0       & -4    & 1 \\ 
      4       & 0    & -3 \\
      -1       & 3     & 0 \\
      
    \end{pmatrix} \begin{pmatrix}
      1\\ 
      -1\\
      1\\
      
    \end{pmatrix} = \begin{pmatrix}
      5\\ 
      1\\
      -4\\
      
    \end{pmatrix} 
\end{aligned}
\end{equation}
%\text{or, } $[\vec{a}]_x \vec{b}$ = 5\^i + \^j - 4\^k.\\
Now, $\abs{[\vec{a}]_x \vec{b}} $ = $\sqrt{5^2 + 1^2 + (-1)^2}$\\
\text{or, } $\abs{[\vec{a}]_x \vec{b}} $ = $\sqrt{27}$\\
\text{or, } $Area$= $\abs{\vec{a} \Cross \vec{b}} $ = $\abs{[\vec{a}]_x \vec{b}} $ = $\sqrt{27}$\\
Hence, the area of the parallelogram in the above problem statement is $\sqrt{27}$.\\